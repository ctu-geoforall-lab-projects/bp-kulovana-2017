\chapter{Závěr}
\label{5-zaver}

%Na čem se dá ještě dál pracovat:
%generalizace
%interpolace
%zpracování bodových dat (ne rastru)

Tato bakalářská práce si kladla za cíl tvorbu zásuvného modulu
automatizujícího zpracování dat získaných v rámci terénního radiačního
průzkumu a jeho implementaci do open source prostředí QGIS. Požadavek
na tvorbu tohoto softwarového nástroje vznesla Armáda ČR, jejíž
operátoři dosud zpracovávali naměřená data ručně. 
% Práce byla doplněna o teoretickou část s úmyslem seznámit čtenáře se zá\-kladními informacemi o radiačním průzkumu.
%%% TK: odstraněno, ne tak důležité a pak se závěr vejde na 1 A4

V současnosti zásuvný modul umožňuje ze vstupní interpolované mapy
dávkového příkonu či plošné aktivity vytvořit izolinie v uživatelem
navolených úrovních. Ty jsou následně převedeny na polygony. Poté je
provedena generalizace jednotlivých polygonů na max. počet bodů = 50
(hodnota vychází z požadavků na výstup). Ty jsou převedeny do
souřadnic systému \zk{MGRS} a vypsány do textového reportu
kompatibilního s formátem dle katalogu APP-11. Uživatel si může sám
zvolit, zda chce vytvořit soubor s grafickým znázorněním polygonů ve
formátu Esri Shapefile. Nástroj i návod k jeho použití jsou v
anglickém jazyce, protože v případě, že se osvědčí, existuje možnost
jeho využití i v zahraničních institucích.

Při tvorbě zásuvného modulu nastalo několik situací, kdy bylo nutno
změnit zvolený postup či kontaktovat zadavatele pro upřesnění
informací. Tím se zpracování prodloužilo, kvůli čemuž u zjednodušených
polygonů prozatím není ošetřeno, aby se vzájemně neprotínaly. Tyto
úpravy jsou pro správnou funkčnost zásuvného modulu dle zadání nutné,
ale zároveň je jejich implementace časově náročná. Autorka na~nich
bude pracovat po odevzdání práce.

Jako další vylepšení modulu v budoucnosti se nabízí rozšíření
vstupních formátů o naměřená bodová data a jejich následnou
interpolaci (pravděpodobně s využitím SAGA-GIS nebo GRASS GIS
\zk{API}). Studiu těchto postupů se autorka věnovala v počátcích
práce, kdy se ještě uvažovalo o zapracování této funkce do zásuvného
modulu v rámci bakalářské práce. S ohledem na časovou náročnost od
implementace bylo upuštěno, ale je možnost se k ní vrátit, až bude
modul plně funkční.

Zásuvný modul využívá knihoven QGIS a musí být distribuován pod
stejnou licencí, tedy GNU \zk{GPL}. Modul je volně dostupný v
repozitáři CTU GeoForAll
Lab\footnote{\url{https://github.com/ctu-geoforall-lab-projects/bp-kulovana-2017}},
zde jsou uveřejněna i testovací data. Po jeho dokončení bude začleněn
do~oficiálního QGIS repozitáře.
