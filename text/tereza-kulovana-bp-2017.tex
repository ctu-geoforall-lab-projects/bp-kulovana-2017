%%%%%%%%%%%%%%%%%%%%%%%%%%%%%%%%%%%%%%%%%%%%%%%%%%%%%%%%%%%%%%%%%%%%%%%%%%%%%%%%%%%%%%%%%%%%%%%%%%%%%%%
%%													%%
%% 	BAKALÁŘSKÁ PRÁCE -  Zásuvný modul QGIS pro terénní radiační průzkum			%%
%% 				 Tereza Kulovaná							%%
%%													%%
%% pro formátování využita šablona: http://geo3.fsv.cvut.cz/kurzy/mod/resource/view.php?id=775 	%%
%%													%%
%%%%%%%%%%%%%%%%%%%%%%%%%%%%%%%%%%%%%%%%%%%%%%%%%%%%%%%%%%%%%%%%%%%%%%%%%%%%%%%%%%%%%%%%%%%%%%%%%%%%%%% 

\documentclass[%
  12pt,         			% Velikost základního písma je 12 bodů
  a4paper,      			% Formát papíru je A4
  oneside,       			% Oboustranný tisk
  pdftex,				    % překlad bude proveden programem 'pdftex' do PDF
%%%  draft
]{report}       			% Dokument třídy 'zpráva'
%

\newcommand{\Fbox}[1]{\fbox{\strut#1}}

\usepackage[czech, english]{babel}	% použití češtiny, angličtiny
\usepackage[utf8]{inputenc}		% Kódování zdrojových souborů je UTF8

\usepackage[square,sort,comma,numbers]{natbib}

\usepackage{caption}
\usepackage{subcaption}
\captionsetup{font=small}
\usepackage{enumitem} 
\setlist{leftmargin=*} % bez odsazení

\makeatletter
\setlength{\@fptop}{0pt}
\setlength{\@fpbot}{0pt plus 1fil}
\makeatletter

\usepackage[dvips]{graphicx}   
\usepackage{color}
\usepackage{transparent}
\usepackage{wrapfig}
\usepackage{float} 

\usepackage{cmap}           
\usepackage[T1]{fontenc}    

\usepackage{textcomp}
\usepackage[compact]{titlesec}
\usepackage{amsmath}
\addtolength{\jot}{1em} 

\usepackage{chngcntr}
\counterwithout{footnote}{chapter}

\usepackage{acronym}

\usepackage[
    unicode,                
    breaklinks=true,        
    hypertexnames=false,
    colorlinks=true, % true for print version
    citecolor=black,
    filecolor=black,
    linkcolor=black,
    urlcolor=black
]{hyperref}         

\usepackage{url}
\usepackage{fancyhdr}
%\usepackage{algorithmic}
\usepackage{algorithm}
\usepackage{algcompatible}
\renewcommand{\ALG@name}{Pseudokód}% Update algorithm name
\def\ALG@name{Pseudokód}

\usepackage[
  cvutstyle,          
  bachelor           
]{thesiscvut}


\newif\ifweb
\ifx\ifHtml\undefined % Mimo HTML.
    \webfalse
\else % V HTML.
    \webtrue
\fi 

\renewcommand{\figurename}{Obrázek}
\def\figurename{Obrázek}

%%%%%%%%%%%%%%%%%%%%%%%%%%%%%%%%%%%%%%%%%%%%%%%%%%%%%%%%%%%%%%%%%
%%%%%%%%%%% Definice informací o dokumentu  %%%%%%%%%%%%%%%%%%%%%
%%%%%%%%%%%%%%%%%%%%%%%%%%%%%%%%%%%%%%%%%%%%%%%%%%%%%%%%%%%%%%%%%

%% Název práce
\nazev{Zásuvný modul QGIS pro~terénní radiační průzkum}
{Radiation Reconnaissance Results to MGRS-described Polygon QGIS Plugin}

%% Jméno a příjmení autora
\autor{Tereza}{Kulovaná}

%% Jméno a příjmení vedoucího práce včetně titulů
\garant{Ing.~Martin~Landa,~Ph.D.}

%% Označení programu studia
\programstudia{Geodézie a~kartografie}{}

%% Označení oboru studia
\oborstudia{Geodézie, kartografie a~geoinformatika}{}

%% Označení ústavu
\ustav{Katedra geomatiky}{}

%% Rok obhajoby
\rok{2017}

%Mesic obhajoby
\mesic{červen}

%% Místo obhajoby
\misto{Praha}

%% Abstrakt
\abstrakt{Tato bakalářská práce si klade za cíl zautomatizovat zpracování dat naměřených 
při~terénním radiačním průzkumu. Z mapy dávkových příkonů, resp. plošné 
aktivity, softwarový nástroj vygeneruje zjednodušené polygony ohraničující 
oblasti dle zvolených úrovní a textový report kompatibilní s~formátem dle NATO 
APP-11. Požadavek na~tento nástroj vzešel ze~strany Armády České republiky, 
jelikož v~době zadání práce celý postup prováděl operátor ručně. Nástroj je 
s~využitím externí knihovny implementován do~prostředí open source systému 
QGIS.} {The aim of this bachelor thesis is to automate processing of data measured 
during the radiation reconnaisance. From map of dose rate or surface 
activity software tool generates simplified polygons bounding areas of preset 
levels and a text file in NATO APP-11 compatible format. This plugin was 
created for Army of the Czech Republic because in time of thesis 
assignment operator had to process data manually. Plugin uses an external 
library and is implemented into open source project QGIS.}

%% Klíčová slova
\klicovaslova
{QGIS, zásuvný~modul, Python, GDAL, radiace}
{QGIS, plugin, Python, GDAL, radiation}

%%%%%%%%%%%%%%%%%%%%%%%%%%%%%%%%%%%%%%%%%%%%%%%%%%%%%%%%%%%%%%%%%%%%%%%%

%%%%%%%%%%%%%%%%%%%%%%%%%%%%%%%%%%%%%%%%%%%%%%%%%%%%%%%%%%%%%%%%%%%%%%%%
%% Nastavení polí ve Vlastnostech dokumentu PDF
%%%%%%%%%%%%%%%%%%%%%%%%%%%%%%%%%%%%%%%%%%%%%%%%%%%%%%%%%%%%%%%%%%%%%%%%
\nastavenipdf
%%%%%%%%%%%%%%%%%%%%%%%%%%%%%%%%%%%%%%%%%%%%%%%%%%%%%%%%%%%%%%%%%%%%%%%

%%% Začátek dokumentu
\begin{document}

\catcode`\-=12  % pro vypnuti aktivniho znaku '-' pouzivaneho napr. v \cline 

% aktivace záhlaví
\zahlavi

% předefinování vzhledu záhlaví
\renewcommand{\chaptermark}[1]{%
	\markboth{\MakeUppercase
	{%
	\thechapter.%
	\ #1}}{}}

% Vysázení přebalu práce
%\vytvorobalku

% Vysázení titulní stránky práce
\vytvortitulku

% Vysázení listu zadani
\stranka{}%
	{\includegraphics[scale=0.7]{./pictures/zadani.pdf}}%\sffamily\Huge\centering\ }%ZDE VLOŽIT LIST ZADÁNÍ}%
	%{\sffamily\centering Z~důvodu správného číslování stránek}

% Vysázení stránky s abstraktem
\vytvorabstrakt

% Vysázení prohlaseni o samostatnosti
\vytvorprohlaseni

% Vysázení poděkování
\stranka{%nahore
       }{%uprostred
       }{%dole
       \sffamily
	\begin{flushleft}
		\large
		\MakeUppercase{Poděkování}
	\end{flushleft}
	\vspace{1em}
		%\noindent
	\par\hspace{2ex}
	{V~první řadě děkuji vedoucímu bakalářské práce, Ing. Martinu Landovi, PhD., za~připomínky a pomoc při~zpracování této práce. Dále děkuji Mgr. Janu Helebrantovi (SÚRO) a nadporučíkovi Jiřímu Komárkovi (AČR) za~odborné rady. V~neposlední řadě děkuji svým blízkým za~projevenou podporu.}
}

% Vysázení obsahu
\obsah

% Vysázení seznamu obrázků
\seznamobrazku

% Vysázení seznamu tabulek
\seznamtabulek

% jednotlivé kapitoly
\chapter{Úvod}
\label{1-uvod}

% Terénní radiační průzkum - co to je?
% Proč armáda poptává plugin?
% Do čeho se to má implementovat?
% Jaké programy a nástroje jsou uvažovány?
% Co mě osobně k tomu vedlo?

% Hlavním cílem terénního radiačního průzkum je zmapování úrovně radiace na území zasaženém havárií a co nejrychlejší předání naměřených výsledků ve vhodném formátu dalším navazujícím složkám. 


(Potenciální) hrozba jaderného výbuchu či jaderné havárie se v posledních sto letech stala více než reálnou. Proto ve světě vznikla potřeba být na takovéto situace co nejlépe připraven. V první řadě existuje samozřejmě snaha jim předcházet, avšak pokud již některý ze zmíněných stavů nastane, je důležité na něj reagovat rychle a efektivně. 
Důsledkem jsou snahy o zjednodušení získávání informací, automatizaci jejich zpracování a standardizaci formátu, v němž jsou předávány navazujícím složkám soustavy. Jedním ze způsobů, jak do tohoto obrovského a provázaného systému přispět, je i softwarový nástroj, jehož vytvoření v rámci této bakalářské práce zadala Armáda České republiky, přesněji 314. centrum výstrahy proti zbraním hromadného ničení v Hostvici-Břve.

314. centrum výstrahy \zk{ZHN} je podřízeno 31. pluku radiační, chemické a biologické ochrany v Liberci. V Armádě ČR plní funkci související se sledováním a vyhodnocováním informací v oblasti radiační, chemické a biologické ochrany.

V období míru je úkolem 314. centra mimo jiné spravovat armádní radiační monitorovací síť Armády České republiky, provádět letecký radiační průzkum, shromažďovat informace o zbraních hromadného ničení, jaderných energetických zařízeních a navrhovat ochranná opatření proti \zk{ZHN} či ochranu proti následkům radiačních havárií. "Při vyhlášení stavu ohrožení státu/válečného stavu přebírá centrum od Ministerstva vnitra úkoly ústředního koordinačního orgánu v oblasti monitorování a výstrahy". (1)

V současnosti AČR zpracovává hodnoty naměřené v rámci radiačního průzkumu ručně, za výpočty a zakreslení výsledků do mapy je zodpovědná analytická skupina. Tento proces je poměrně náročný na znalosti a zkušenosti operátora, ani čas strávený vyhodnocením není zanedbatelný. Výhodný by proto byl softwarový nástroj, který by část procesu zautomatizoval. Konkrétně se jedná o vytvoření předdefinovaných izolinií ze vstupního interpolovaného gridu, jejich převod na zjednodušené polygony a vygenerování textového reportu ve formátu dle specifikace \zk{NATO}/\zk{AČR} v souřadnicích v systému \zk{MGRS}. Na operátorovi pak zůstane následné vložení zprávy do softwaru určeného k varování a uvědomování ostatních jednotek. 

S ohledem na skutečnost, že \zk{AČR} zpracovává data v open source geografickém informačním systému QGIS, bylo rozhodnuto, že nástroj bude vyvíjen jako nový zásuvný modul pro toto prostředí. Modul bude psán v programovacím jazyku Python, pro grafické rozhraní bude použit framework Qt, bude využívat QGIS \zk{API}. Pro své specializované funkce je uvažováno i využití GRASS GIS \zk{API} a knihovna \zk{GDAL}. 

V teoretické části práce bude čtenář seznámen se způsoby monitorování radiační situace v České republice, dotkne se tématu standardizovaných zpráv předávaných v rámci armády a představí hlásný systém \zk{MGRS}.
     
% (1) http://www.acr.army.cz/scripts/detail.php?id=9473
\chapter{Teoretický základ}
\label{2-teorie}

\section{Sběr dat}


\section{Textový report}


\subsection{MGRS}
Military grid reference system je zeměpisný souřadnicový systém, který umožňuje vyjadřovat polohu bodů na zemském povrchu v alfanumerických znacích. (zdroj: https://boundlessgeo.com/2015/04/mgrs-coordinates-qgis/)
Narozdíl od jiných souřadnicových systémů, které vyjadřují polohu pomocí dvojice hodnot (šírka, délka, případně x, y), MGRS využívá jen jednu hodnotu.



\chapter{Použité technologie}
\label{3-technologie}

Třetí kapitola popisuje jednotlivé technologie použité při tvorbě zásuvného modulu.

\section{QGIS}

\begin{figure}[H]
    \centering
      \includegraphics[width=100pt]{./pictures/qgis-logo.png}
      \caption[QGIS logo]{QGIS logo 
      (zdroj: \href{https://www.qgis.org/en/_downloads/qgis-logo.png}{QGIS})}
      \label{fig:qgis}
  \end{figure}

QGIS je multiplatformní volně dostupný geografický informační systém (\zk{GIS}).
Každý, kdo bude chtít nějakým způsobem přispět, je vítán.


\section{Python}

\begin{figure}[H]
    \centering
      \includegraphics[width=100pt]{./pictures/python-logo-master-v3-TM.png}
      \caption[Python logo]{Python logo 
      (zdroj: \href{https://www.python.org/static/community_logos/python-logo-master-v3-TM.png}{Python.org})}
      \label{fig:python}
  \end{figure}
  
\subsection{GDAL}  

\section{Qt Project}

\begin{figure}[H]
    \centering
      \includegraphics[width=100pt]{./pictures/qt-logo-small.png}
      \caption[Qt Project logo]{Qt Project logo 
      (zdroj: \href{http://s3-eu-west-1.amazonaws.com/qt-files/logos/Qt-logo-small.png}{qt.io})}
      \label{fig:qt}
  \end{figure}

\subsection{PyQt}
\chapter{Zásuvný modul}
\label{4-plugin}

\section{Vstupní data}

\section{Tvorba izolinií}

\subsection{Uzavírání linií}

\section{Generalizace}

\section{Tvorba polygonů}

\section{Výstupní report}
\chapter{Závěr}
\label{5-zaver}

Na čem se dá ještě dál pracovat:
generalizace
interpolace
zpracování bodových dat (ne rastru)


% Vysázení seznamu zkratek

\begin{seznamzkratek}{ABCDE}

	\novazkratka{SÚRO}
	      {SÚRO}
	      {Státní ústav radiační ochrany}
	      
	 \novazkratka{AČR}	
	     {AČR}
	     {Armáda České republiky (Army of the Czech republic)}	      
	      
	\novazkratka{PSF}
		  {PSF}
	      {Python Software Foundation}

	\novazkratka{GIS}
	      {GIS}
	      {Geografický informační systém (Geographic information system)}

      \novazkratka{OSGeo}
	      {OSGeo}
	      {Open Source Geospatial Foundation}
	         
	  \novazkratka{GUI}	
	      {GUI}
	      {Grafické uživatelské rozhraní (Graphical user interface)}
	           
	  \novazkratka{WGS84}	
	      {WGS84}
	      {Světový geodetický systém 1984 (World Geodetic System 1984)}

	  \novazkratka{GPL}	
	      {GPL}
	      {Všeobecná veřejná licence (General Public License)}
	      
	  \novazkratka{GDAL}	
	      {GDAL}
	      {Všeobecná veřejná licence (Geospatial Data Abstraction Library)}
	    
	  \novazkratka{MGRS}	
	      {MGRS}
	      {Military grid reference system}
	      
	  \novazkratka{NATO}	
	      {NATO}
	      {}
	      
	  \novazkratka{UTM}	
	      {UTM}
	      {Univerzální transverzální Mercatorův systém souřadnic (Universal Transverse Mercator)}	
	      
	  \novazkratka{UPS}	
	      {UPS}
	      {(Universal polar stereographic)}	            	      

\end{seznamzkratek}

% Literatura
\nocite{*}
\def\refname{Literatura}
\bibliographystyle{mystyle}
\bibliography{literatura}


% Začátek příloh
\def\figurename{Figure}%
\prilohy

% Vysázení seznamu příloh
%\seznampriloh

% Vložení souboru s přílohami
\include{prilohy}

% Konec dokumentu
\end{document}
