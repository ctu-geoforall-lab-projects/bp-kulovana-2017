\chapter{Úvod}
\label{1-uvod}

Terénní radiační průzkum - co to je?
Proč armáda poptává plugin?
Do čeho se to má implementovat?
Jaké programy a nástroje jsou uvažovány?
Co mě osobně k tomu vedlo?

Využití GRASS GIS API nebo knihovny GDAL, spíš ten GRASS. 

Hlavním cílem terénního radiačního průzkum je zmapování úrovně radiace na území zasaženém havárií a co nejrychlejší předání naměřených výsledků ve vhodném formátu dalším navazujícím složkám. 


(Potenciální) hrozba jaderného výbuchu či jaderné havárie se v posledních sto letech stala více než reálnou. Proto ve světě vznikla potřeba být na takovéto situace co nejlépe připraven. V první řadě existuje samozřejmě snaha jim předcházet, avšak pokud již některý ze zmíněných stavů nastane, je důležité na něj reagovat rychle a efektivně. 
Důsledkem jsou snahy o zjednodušení získávání informací, automatizaci jejich zpracování a standardizaci formátu, v němž jsou předávány navazujícím složkám soustavy. Jedním ze způsobů, jak do tohoto obrovského a provázaného systému přispět, je i softwarový nástroj, jehož vytvoření v rámci této bakalářské práce zadala Armáda České republiky.
Úkolem 314. centre výstrahy zbraní hromadného ničení Hostivice-Břve 


je monitorovat a vyhodnocovat stav prostředí v úvarech a zařízeních Armády České republiky s důrazem na látky naznačující použití zbraní hromadného ničení.

Centrum je součástí 31. pluku radiační, chemické a biologické ochrany Liberec. Plní odborné úkoly související se sledováním a vyhodnocováním informací v oblasti radiační, chemické a biologické ochrany. 
