\chapter{Použité technologie}
\label{3-technologie}

Třetí kapitola popisuje jednotlivé technologie použité při tvorbě zásuvného modulu.

\section{QGIS}

\begin{figure}[H]
    \centering
      \includegraphics[width=100pt]{./pictures/qgis-logo.png}
      \caption[QGIS logo]{QGIS logo 
      (zdroj: \href{https://www.qgis.org/en/_downloads/qgis-logo.png}{QGIS})}
      \label{fig:qgis}
  \end{figure}

QGIS je multiplatformní volně dostupný geografický informační systém (\zk{GIS}).
Každý, kdo bude chtít nějakým způsobem přispět, je vítán.


\section{Python}

\begin{figure}[H]
    \centering
      \includegraphics[width=100pt]{./pictures/python-logo-master-v3-TM.png}
      \caption[Python logo]{Python logo 
      (zdroj: \href{https://www.python.org/static/community_logos/python-logo-master-v3-TM.png}{Python.org})}
      \label{fig:python}
  \end{figure}
  
\subsection{GDAL}  

\section{Qt Project}

\begin{figure}[H]
    \centering
      \includegraphics[width=100pt]{./pictures/qt-logo-small.png}
      \caption[Qt Project logo]{Qt Project logo 
      (zdroj: \href{http://s3-eu-west-1.amazonaws.com/qt-files/logos/Qt-logo-small.png}{qt.io})}
      \label{fig:qt}
  \end{figure}

\subsection{PyQt}