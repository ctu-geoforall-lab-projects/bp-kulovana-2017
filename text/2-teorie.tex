\chapter{Teoretický základ}
\label{2-teorie}

V této kapitole jsou objasněny způsoby monitorování radiační situace, popsán sběr dat a představen výstupní report.

\section{Radiační průzkum}
Systematické plošné monitorování radiační situace má na území České republiky počátky v dubnu 1986 (1), kdy došlo k havaráii v JE Černobyl. Je zajišťováno především pomocí celostátní Radiační monitorovací sítě (\zk{RMS}). 

RMS běžně operuje v tzv. normálním režimu, v případě mimořádné radiační situace přechází do tzv. havarijního režimu. Během normálního režimu pracují stálé složky RMS, jež v první řadě zajišťují provoz fixních měřících míst a analýzu dat z nich získaných. Mezi stálé složky RMS patří Státní ústav pro jadernou bezpečnost {\zk(SÚJB)}, Státní ústav radiační ochrany {\zk(SÚRO)} a Český hydrometeorologický úřad {\zk(ČHMÚ)}. Při přechodu do havarijního režimu dochází rovněž k aktivaci pohotovostních složek. Po zvážení je zahájen radiační průzkum, který sestává z dalšího monitorování na měřících bodech, pojezdového měření a v případě potřeby i měření leteckého.

% (1) http://www.mocr.army.cz/informacni-servis/zpravodajstvi/chemici-se-specialisty-statniho-ustavu-radiacni-ochrany-spolecne-monitorovali-radiaci-z-vrtulniku-118918/
- něco o jednotkách a veličinách



\subsection{Sběr dat}

Za měření radioaktivity během radiačního průzkumu je u Armády České republiky zodpovědné především 314. centrum výstrahy proti zbraním hromadného ničení v Hostivici-Břve, které je podřízeno 31. pluku radiační, chemické a biologické ochrany v Liberci. 
 
	Konkrétní typy dopravních prostředků a měřících přístrojů zmíněné v textu níže platí pro 314. centrum výstrahy proti ZHN, které poskytlo testovací data pro vývoj zásuvného modulu a které jej bude především využívat. 

Typy měření:
\begin{itemize}
	\item \textbf{napevno umístěné detektory (měřící body)}
	
	Napevno umístěné detektory jsou v činnosti nepřetržitě. Mezi hlavní sítě na území ČR patří Síť včasného zjištění se 71 měřícími místy spravovaná Regionálními centry (RC) {\zk(SÚJB)}, {\zk(SÚRO)}, {\zk(ČHMÚ)}, Hasičským záchranným sborem ČR a Armádou ČR a teritoriální síť TLD, tvořená 185 měřícími místy (2). Vedle nich existují i lokální sítě monitorující okolí JE Dukovany a JE Temelín.
	Během havarijního režimu hodnoty získané z měřících bodů doplňují informace z terénního průzkumu. 
	
% (2) https://www.suro.cz/cz/rms	
	
	\item \textbf{pojezdové měření}
	
	Sběr dat při pojezdovém měření je prováděn z vozidla Land Rover LR-110, lehkého obrněného kolového transportéru BRDM-2 nebo džípu UAZ-469 (postupně vyřazován). Přístroje používané k měření dávkového příkonu jsou DP-98, AS-67 nebo DP-3b a jsou pevně spojené s vozidlem. 
	Při naměření předem stanoveného dávkového příkonu osádka vozu již dále nepokračuje ve směru rostoucích hodnot do epicentra výbuchu. "Takováto úroveň se pouze vytyčí a souřadnice jejího naměření se zahlásí radiostanicí na sběrné stanoviště (veliteli jednotky radiačního průzkumu, popřípadě na analytickou skupinu)," popisuje nadporučík Jiří Komárek, starší důstojník Skupiny monitorování a leteckého průzkumu 314. centra výstrahy ZHN. Následně se osádka vrací zpět po stejné trase. 
	Právě závislost pozemního průzkumu na trasách přesunu, tj. cestách, je jeho největší nevýhodou. Proto je vhodné ho ve složitějším terénu či při velké havárii doplnit měřením leteckým.
	
	\item \textbf{letecké měření z vrtulníku}
	
	Oproti pojezdovému měření si letecký průzkum může dovolit prozkoumat kontaminovaný prostor více do hloubky. Úkolem specialistů na palubě však je sledování měřených dávkových příkonů, aby eventuálně mohli buď upravit parametry průzkumu (rychlost, výška), nebo změnit trasu letu. Cílem je zmapování velké oblasti bez ohledu na charakter terénu. 
	"V případě jaderného výbuchu by mohl být letecký průzkum potenciálně využit ke zmapování radioaktivní stopy, kterou takový výbuch po vypadání částic zanechá. Důležité jsou tady samozřejmě vhodně zvolené podmínky průzkumu," doplňuje dále nadporučík Komárek. 
	Průzkum nemůže proběhnout do chvíle, než vypadají radioaktivní částice na zemský povrch. Během čekání lze předběžně určit orientační dávkové příkony v epicentru vztažené na odhad mohutnosti výbuchu, na jejichž základě se rozhodne o provedení průzkumu ve vhodném časovém horizontu (např. po "vymření" krátkodobých radionuklidů, kdy radiace v epicentru poklesne). Následně lze provést průzkum v souladu s principem ALARA, tedy že dávka ionizujícího záření, které je osoba vystavena, má být tak nízká, jaké lze rozumně dosáhnout.
	
	K provedení leteckého radiačního průzkumu v současné době disponujeme systémem IRIS (velkoobjemový NaI(Tl) doplněný o GM trubici pro velké četnosti záření), přístrojem MobDOSE a palubním detektorem DP-3a.

	Výstup z IRISu a MobDOSE poskytne georeferencovaná data o dávkovém příkonu (PFDE), ale je zase na operátorovi letecké skupiny, aby tato data zanesl ručně do softwaru tak, aby mohl být výstup distribuován ostatním jednotkám formou standardizované CBRN zprávy.

	\item \textbf{měření bezpilotními prostředky}
\end{itemize}

	Zakreslování do mapy opravdu probíhá ručně, za to je zodpovědná právě analytická skupina. Informace mohou být dále zaneseny do softwaru určeného k varování a uvědomování jednotek o incidentech zahrnujících zbraně hromadného ničení (ZHN) a průmyslové nebezpečné látky (PNL), v souhrnné zkratce CBRN. 
	
\subsection{Měřené veličiny}




\section{Textový report}


\subsection{Military grid reference system}
Hlásný systém MGRS (\zk{MGRS}) je systém udávání polohy používaný Severoatlantickou aliancí (\zk{NATO}). Využívá Mercatorovo příčné válcové konformní zobrazení (\zk{UTM}), případně UPS. 

Na rozdíl od jiných souřadnicových systémů, které vyjadřují polohu pomocí dvojice hodnot (šířka/délka, x/y), MGRS využívá jen jednu hodnotu a to alfanumerický řetězec znaků. Ten je tvořen třemi údaji:

\begin{itemize}
	\item \textbf{označení zóny}
	
	Jedna zóna je tvořena sférickým čtyřúhelníkem referenčního elipsoidu, jenž je vymezen zeměpisnými poledníky a rovnoběžkami. Sférické čtyřúhelníky vznikají rozdělením povrchu Země do 60 poledníkových zón o šířce 6°, které jsou následně děleny ve směru rovnoběžek na 19 vrstev po 8° a 1 vrstvu o výšce 12°.
	
	Poledníkové pásy jsou číslovány od 1 do 60 od obrazu poledníku 180° z. d. směrem na východ. Vrstvy jsou značeny velkými písmeny latinské abecedy C-X (s vynecháním písmen I a O vzhledem k jejich podobnosti s číslicemi) od obrazu rovnoběžky 80° j. š. na sever.
	
	Jedinečné označení zóny je složeno z čísla poledníkového pásu následovaného písmenem rovnoběžkového pásu (např. 33U).
	
	\item \textbf{označení čtverce 100 x 100 km}
	
	Jednotlivé zóny jsou rozděleny na čtverce o hraně 100 km sítí čar rovnoběžných s obrazem příslušného osového poledníku a rovníku. Jelikož se poledníkové pásy směrem k pólům zužují, zóny obsahují určitý počet úplných čtverců a na krajích neúplné čtverce o proměnlivé šířce. 
	
	Pro označení sloupců jsou použita písmena A-Z (s vynecháním I a O), značení začíná u obrazu poledníku 180° z. d. a pokračuje směrem na východ, po písmenu Z se celá řada opět opakuje. Vrstvám jsou přidělena písmena A-V (bez I a O). První vrstva lichých poledníkových pásů je značena písmenem A, u sudých pásů začíná písmenem F. Po písmenu V se abeceda opakuje. 
	
	Označení čtverce se skládá ze dvou písmen - označení sloupce a vrstvy (např. VR)
		
	\item \textbf{souřadnice bodu ve 100 km čtverci}
	
	V rámci čtverce je upřesněna poloha bodu za pomoci n+n číslic, kde první sada číslic určuje východní souřadnici od levého kraje čtverce a druhá sada severní souřadnici od okraje spodního. Podle přesnosti vyjádření polohy bodu n nabývá hodnot 1, 2, 3, 4 nebo 5. 
	
		\begin{itemize}
				\item 1+1 číslice pro souřadnici s přesností 10 km (\textit{54})
				\item 2+2 číslice pro souřadnici s přesností 1 km (\textit{5748})
				\item 3+3 číslice pro souřadnici s přesností 100 m (\textit{577484})
				\item 4+4 číslice pro souřadnici s přesností 10 m (\textit{57704840})
				\item 5+5 číslic pro souřadnici s přesností 1 m (\textit{5770048400})
		\end{itemize}	
		 
\end{itemize}

Poloha Fakulty stavební ČVUT v Praze by tedy pomocí hlásného systému MGRS s přesností na metry byla vyjádřena řetězcem ...

Standardem NATO je rozlišení 10 m [wiki], Armáda ČR pro výstup zásuvného modulu požaduje přesnost na 1 m.

Východní a severní souřadnice v systému MGRS se vždy vztahují k levému dolnímu rohu čtverce. Při přechodu na nižší přesnost se souřadnice nezaokrouhlují, ale přebytečné číslice se odříznou, aby bylo zajištěno, že bod zůstane ve správném čtverci s nižší přesností.

% Souřadnice (například 33UVR577484, tj. 33U VR 577 484) se skládají z několika informací:



%zdroje: 
% https://boundlessgeo.com/2015/04/mgrs-coordinates-qgis/ - vlastně vůbec
% https://www.vugtk.cz/slovnik/5479_hlasny-system-mgrs - základní definice
% http://uhulag.mendelu.cz/files/pagesdata/cz/geodezie/geodezie1/souradnicove_systemy.pdf (str. 48)
% http://www.diverzanti.cz/cl_36a - nejvíc
% https://en.wikipedia.org/wiki/Military_Grid_Reference_System