\chapter{Teoretický základ}
\label{2-teorie}

\section{Sběr dat}


\section{Textový report}


\subsection{MGRS}
Hlásný systém MGRS (\zk{MGRS}) je systém udávání polohy používaný v \zk{NATO}. Využívá Mercatorovo příčné válcové konformní zobrazení v šestistupňových poledníkových pásech (\zk{UTM}) a UPS. 


který umožňuje vyjadřovat polohu bodů na zemském povrchu v alfanumerických znacích. 
Na rozdíl od jiných souřadnicových systémů, které vyjadřují polohu pomocí dvojice hodnot (šírka, délka, případně x, y), MGRS využívá jen jednu hodnotu.


zdroje: 
% https://boundlessgeo.com/2015/04/mgrs-coordinates-qgis/
% https://www.vugtk.cz/slovnik/5479_hlasny-system-mgrs
% http://uhulag.mendelu.cz/files/pagesdata/cz/geodezie/geodezie1/souradnicove_systemy.pdf (str. 48)